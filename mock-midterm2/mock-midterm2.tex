\documentclass{article}

\usepackage{amsmath}
\usepackage[utf8]{inputenc}
\usepackage[spanish, mexico]{babel}
\usepackage[margin=0.5in]{geometry}
\usepackage{listings}

\lstset{
	columns=fullflexible,
	frame=single,
	breaklines=true,
}

\title{Mock Midterm	2}
\author{Miguel A. Gomez B.}

\begin{document}
	\maketitle
	
\paragraph{1} List the matrix form of the numerical solver techniques for solving the heat diffusion equation.
\paragraph{Solution}
Given that:
\begin{itemize}
	\item $I$ is the identity matrix.
	\item $A$ \textbf{is another matrix...}
	\item $\Delta t$ is the simulation parameter for time.
	\item $a$ is the diffusion coefficient.
	\item $h$ is the simulation parameter for an one-dimensional space rod.
	\item $\overrightarrow{\rm U}_0$ is an initial solution of the ODE.
\end{itemize}
\textbf{The explicit solver:}
$$ \overrightarrow{\rm U}^{n+1} = \left(I + A\frac{\Delta t a}{h^2}\right)^{n+1} \cdot \overrightarrow{\rm U}_0$$
\textbf{The implicit solver:}
$$\overrightarrow{\rm U}^{n+1} = \left(I - A\frac{\Delta t a}{h^2}\right)^{-1} \cdot \overrightarrow{\rm U}_{0}$$
\textbf{The Crank-Nicolson solver:}
$$\overrightarrow{\rm U}^{n+1} = \left[\left(I + A\frac{\Delta t a}{h^2}\right) \cdot \left(I - A \frac{\Delta t a}{h^2}\right)^{-1} \right]^{n+1} \cdot \overrightarrow{\rm U}_{0}$$
\paragraph{2} Write a program to use the explicit method to solve the diffusion equation
\begin{align*}
	u_t &= u_{xx}, t > 0, 0 < x < 1;\\
	u(0, t) &= 0; \\
	u(1, t) &= 0; \\
	u(x, 0) &= \sin{(\pi x)}.
\end{align*}
which has the exact solution:
$$u(x, t) = e^{-\pi^2 t} \sin{(\pi x)}$$
(the student should check that this is indeed the exact solution). Use $h^{-1} = 4, 8, \dots, \frac{1}{1024}$, and take $\delta t$ as large as possible to maintain stability. Confirm that the approximate solution is as accurate as the theory predicts. Compute out to $t = 1$.
\paragraph{Solution} We use $x = \frac{1}{16} = 0.0625$. Testing on the solution given $(x,0)$ we obtain on the exact solution formula:
\begin{align*}
	U(0.0625,0) &= e^{-\pi^2 \cdot 0} \sin{(\pi(0.0625)}\\
	&= \sin{(\pi(0.0625))}\\
	&\approx 0.19509032201612825 
\end{align*}
Checking on the initial solution we get:
\begin{align*}
	U(0.0625,0) &= \sin{(\pi(0.0625))}\\
	&= \sin{(\pi(0.0625))}\\
	&\approx 0.19509032201612825 
\end{align*}
which implies it is actually the exact solution. To solve this problem we'll use the Crank-Nicolson solver with a modification to obtain the best solution given that $\Delta t =  \frac{1}{64} = 0.015625$. To test the which parameter will be the best fit to the original equation we will use:

$$\sum_{i=0}^{x^{-1}}$$
\begin{lstlisting}[language=python]
import numpy as np
import numpy.linalg as la
from pprint import pprint

def crank_nicolson(I, dt, a, h, A, n, u_0):
	return np.dot(la.matrix_power(np.matmul(I + (dt*a/h**2)*A ,la.inv(I - A*(a*dt/h**2))), n), u_0)

# Diffusion coefficient
a=np.pi**-2
# Simulation parameters
dt=1/64
h=1/4

b = 4

# Diagonal matrix with -2 in its diagonal.
A = np.diag([-2]*b-1) + np.diag([1]*b-2,-1) + np.diag([1]*b-1,1)

u_0 = np.sin([np.pi*h*i for i in range(1,b)])

I = np.diag([1]*(b-1))

crank_nicolson = crank_nicolson(I, dt, a, h, A, 2, u_0)

pprint(crank_nicolson)
\end{lstlisting}
\paragraph{3} Please load the CVS attached to the midterm and perform constant, linear, cuadratic and cubic regression ¿What regression fits the bests?
\end{document}