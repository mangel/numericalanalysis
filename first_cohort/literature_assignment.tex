\documentclass{article}

\usepackage{amsmath}
\usepackage{amsfonts}
\usepackage{amssymb}
\usepackage[utf8]{inputenc}

\title{Literature review assignment}
\author{Miguel A. Gomez B.}

\begin{document}
	
\maketitle
\paragraph{1} write a paragraph describing the key aspects of the papers listed below, the
paragraph should include:
\begin{itemize}
	\item 2 lines of motivation
	\item 2 lines for problem statement
	\item 2 lines for solution approach
	\item 2 lines for conclusions
\end{itemize}
\paragraph{2} List 10 papers and their abstracts related to the paper. This papers should be cited by the
paper selected above. Papers that cite the paper selected in the previous literal is also
recommended.

\section{Review}

New tools have been discovered to analize biological systems that give better and faster results. But many of these systems are complex, because of the multiple interactions at a macro-scale and micro-scale level. The approach consists on a deep analysis of the existing model, after that, they evaluate and propose the introduction of a stochastic process on the model to benefit of SHS theory. The authors expect run these models on computers: "The hope is to substitute costly and time-consuming in vitro or in vivo experiments or provide guidance for those experiments and generate better hypotheses."\cite{xiangfang_li_review_2017:1}

\section{Papers abstract related to the paper}

\paragraph{Transcriptional stochasticity in gene expression\cite{tomasz_lipniacki_transcriptional_2006}}

The article is focused on highlight and determine that there is an inherent stochasticity in the process of DNA transcription. Some interactions in the model between DNA, mRNA, regulatory proteins and gene expressions must admit stochastic effects. The authors construct a mathematical model to observe and verify these interactions by adding a stochastic component on current partial differential equations. The experimentation and results of the model conclude that there is a close relationship between a small set of mRNA and protein molecules that can be measured for a single cell. It is used on the main article as an evidence of the stochasticty of some biological processes such as DNA transcription.

\paragraph{Diversity in times of adversity: probabilistic strategies in microbial survival games\cite{denise_m._wolf_diversity_2005}}

the article explains that in some cases a probabilistic strategy is adopted by some organism such as \textit{Escherichia Coli} make to survive changes on their environment. The process that triggers this strategy it is too complex to model due to the broad range of changes at a macroscopic and microscopic level. The authors determine that it is better to use the approach from game theory and build a model that uses probabilities to determine wich strategy is used at a given environment. the authors conclude that it is a model that some proof that the alternation between losing strategies produce a wining strategy, but there is also a certain independence between the strategy used, which they assume it is to minimize growth rate at the expense of unfavorable conditions in the environment (the unlucky scenario). It is used on the main article to demonstrate that stochastisity plays an important role on biochemical processes.

\paragraph{In Silico, In Vitro, and In Vivo Studies Indicate the Potential Use of Bolaamphiphiles for Therapeutic siRNAs Delivery\cite{taejin_kim_silico_2013}}

RNA-based therapy technology to treat diseases such as cancer are making progress. Currently some sRNAs are being design to silence the expression of oncogenic paths, but this molecule has a short life on the blood serum, which makes it inefective for cancer treatment. The authors use properties of the Bolaamphiphiles to extend the live of the sRNAs in the blood making it effective for cancer treatment, to verify it, the authors use an arrangement of In silico (computer based), In vitro and In vivo studies that confirm the effectiveness of their method. It is used in the main article to demostrate that computer based model have been effective to analyze and predict results on biochemical reactions.

\paragraph{Intrinsic and extrinsic contributions to stochasticity in gene expression\cite{peter_s._swain_intrinsic_2002}}

It has been probed the stochasticity in gene expression, that the authors afirm are related to the biochemical processes of transcription, translation and the random change of individual cells participating in the reaction. Such randomness can be decomposed into two causes "intrinsic" and "extrinsic" and it is difficult to measure and model, the authors propose a sensitivity analysis to demonstrate the correlation between these causes on the expressability of genes. The authors conclude that there is a strong correlation and tendency related to these causes. It is used in the main article to support the idea that stochasticity is present at a cellular level.

\paragraph{THE BIG CHALLENGES OF BIG DATA\cite{vivien_marx_big_2013}}

There is an increase of the use of computers to analyze complex data between biologists. This data is in the order of petabytes, and with such information many other entities around the world request access to this information, making a colossal challenge to query this data. Several approches are presented: third-party clouds, transfer the data physically, and inventing new protocols outside the internet to improve data transfer.With the current technology it is impossible to replace in vivo and in vitro experiments due to the lack of performance. It is used in the main article to support the idea that biological hypothesis can be tested and verified on computers, if the necessary data is available.

\paragraph{Qualitative Analysis and Verification of Hybrid Models of Genetic Regulatory Networks: Nutritional Stress Response in Escherichia coli\cite{gregory_batt_qualitative_2005}.}

There is a switch like character on the dynamics of genetic regulatory networks that are interesting to mathematical biologists.The authors want to construct a model that describes mathematically this behavior by designing a hybrid system. The model can be computed symbolically. It is used on the main article to support an argument about the mixture between discrete and continuous dynamics on this type of systems.

\paragraph{Hybrid Modeling and Simulation of Genetic Regulatory Networks: A Qualitative approach. \cite{hidde_de_jong_hybrid_2003}}

Recent advances in studies of genetic regulatory networks had allowed new discoveries in gene expression. Better computational tools are needed increase progress to model and simulate these complex systems, the authors propose a new hibrid model that uses piecewise linear differential equations that have been well-studied in the field. The new tool is capable to simulate several networks of biological interest such as the underliying initiation of sporulation in \textit{Bacillus subtilis}. It is used on the main article to support an argument about the current use of hybrid systems.

\paragraph{Modeling and Simulation of Genetic Regulatory Networks.\cite{hidde_de_jong_modeling_2003}}

In recent years due to the arrival of technology to analyze big data sets and the advances on biology, new approaches are needed to analyze biological reactions, on the article the author reviews the traditional approach and the new aproach with stochastic models. It is used in the main article to support the argument that stochastic processes are present on the models that it is intended to be built.

\paragraph{Reconstructing Metabolic Networks Using Interval Analysis.\cite{warwick_tucker_reconstructing_2003}}

Many biologycal models that simulate complex systems are based on ordinary differential equations that involve parameters that are unknown or correspond to certain property of the system, with the availability a new possibility opens to reconstruct these parameters and improve the results of the models, the authors conclude that they approach have better results that other models. It is used in the article to demonstrate the availability of many other approaches to model complex biological systems cannot support the inherent stochasticity that the hybrid models do.

\paragraph{What does biologically meaningful mean? A perspective on gene regulatory network validation.\cite{albertha_jm_walhout_what_2011}}

Many models exists to describe the complex biological process of gene expression, here the author question some existing models that use the continuous approach with the use of multiple differential equations and the discrete approach, that uses logic, network theory to simulate these processes. The author confirms that there are negative implications of using many of these models because miss at some point parts of the phenomena, wich is why she prefers to use at least five levels of validations of these results. It is used on the main article to support an argument related to use only discrete or continuous models intended to propose the hybrid model as the better approach to these kind of problems.

\bibliography{bibliography}
\bibliographystyle{ieeetr}

\end{document}