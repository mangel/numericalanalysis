\documentclass{article}

\usepackage{amsmath}
\usepackage{amsthm}
\usepackage{amssymb}
\usepackage{amsfonts}
\usepackage{longtable}

\title{Mock exam}
\author{Miguel A. Gomez B.}

\begin{document}

\maketitle

\paragraph{1.} For each IVP problem, verify that $y$ solves the problem and describe by euler the approximate solution.

\paragraph{a} $y'+ 4y =0$, $y(0)=3$, $y(t) = 3e^{-4t}$

\subparagraph{} We find the derivative of $y(t)$:

\[
    y' = -12e^{-4t}
\]

Now we substitute back into the original equation and we get:

\begin{align*}
    y'+ 4y &= -12e^{-4t} + 12e^{-4t}\\
    &= 0
\end{align*}

Which is correct.

\subparagraph{} By using Euler numeric method we get the problem:
\[
    \frac{dy}{dt} = -12e^{-4t}
\]
\[
    y(t_0) = 3, t_n = nh, h = 0.2
\]
\[
    y_{n+1} = y_n + hf(t_n,y_n)
\]

the following table contains the results:

\begin{table}[ht]
  \begin{center}
    \label{tab:1.a}
    \begin{tabular}{l|c|r|r} % <-- Alignments: 1st column left, 2nd middle and 3rd right, with vertical lines in between
      $n$&$t_n$&$y_n$&$f(t_n,y_n)$\\
      \hline
        0&0&3&-12\\
        1&0.2&0.6&-5.39195\\
        2&0.4&-0.47839&-2.42276\\
        3&0.6&-0.96294&-1.08862\\
    \end{tabular}
  \end{center}
\end{table}

\newpage

\paragraph{b} $y' = \frac{t^{2}}{y}$, $y(0)=1$, $y(t) = \sqrt{1 + \frac{2}{3}t^{3}}$

\subparagraph{} We find the derivative of $y(t)$:

\[
    y' = \frac{2t^2}{2\sqrt{1+\frac{2}{3}t^{3}}}
\]

Now we substitute back into the original equation and we get:

\begin{align*}
    \frac{2t^2}{2\sqrt{1+\frac{2}{3}t^{3}}} &= \frac{t^2}{\sqrt{1 + \frac{2}{3}t^{3}}}\\
    \frac{2t^2}{ \sqrt{1+\frac{2}{3}t^{3}}}&= \frac{2t^2}{\sqrt{1 + \frac{2}{3}t^{3}}}
\end{align*}

Which is correct.

\subparagraph{} By using Euler numeric method we get the problem:
\[
    \frac{dy}{dt} = \frac{2t^2}{2\sqrt{1+\frac{2}{3}t^{3}}}
\]
\[
    y(t_0) = 1, t_n = nh, h = 0.2
\]
\[
    y_{n+1} = y_n + hf(t_n,y_n)
\]

the following table contains the results:

\begin{table}[ht]
  \begin{center}
    \label{tab:1.b}
    \begin{tabular}{l|c|r|r} % <-- Alignments: 1st column left, 2nd middle and 3rd right, with vertical lines in between
      $n$&$t_n$&$y_n$&$f(t_n,y_n)$\\
      \hline
        0&0&1&0\\
        1&0.2&1&0.03989376\\
        2&0.4&1.007979&0.15669215\\
        3&0.6&1.039317&0.33658092\\
    \end{tabular}
  \end{center}
\end{table}

\newpage

\paragraph{c} $ty' - y = t^{2}$, $y(1)=4$, $y(t)=3t + t^{2}$

\subparagraph{} We find the derivative of $y(t)$:

\[
    y' = 3 + 2t
\]

Now we substitute back into the original equation and we get:

\begin{align*}
    t(3 + 2t) - (3t + t^{2}) &= t^{2}\\
    3t + 2t^{2} - 3t - t^{2} &= t^{2}\\
    t^{2} &= t^{2}
\end{align*}

Which is correct.

\subparagraph{} By using Euler numeric method we get the problem:
\[
    \frac{dy}{dt} = 3 + 2t
\]
\[
    y(t_1) = 4, t_n = nh, h = 0.2
\]
\[
    y_{n+1} = y_n + hf(t_n,y_n)
\]

the following table contains the results:

\begin{table}[ht]
  \begin{center}
    \label{tab:1.c}
    \begin{tabular}{l|c|r|r} % <-- Alignments: 1st column left, 2nd middle and 3rd right, with vertical lines in between
      $n$&$t_n$&$y_n$&$f(t_n,y_n)$\\
      \hline
        1&0.2&4&3.4\\
        2&0.4&4.68&3.8\\
        3&0.6&5.44&4.2\\
        4&0.8&6.28&4.6\\
    \end{tabular}
  \end{center}
\end{table}

\newpage

\paragraph{2.} Compare  Euler's and second order Taylor methods with $h = \frac{1}{4}$ to compute approximate values of $y(1)$ for each of the following initial value problems. Give a table of $(t_k, y_k)$ pairs.

\paragraph{a} $y' = y(1-y)$, $y(0)=1$\newline
To use second order Taylor, we need to find the second derivative with respect of $t$:

\[
    y'' = 1 - 2y
\]

by applying both methods we get the following results:

\begin{table}[ht]
  \begin{center}
    \label{tab:table 2.a}
    \begin{tabular}{|l|c|r|} % <-- Alignments: 1st column left, 2nd middle and 3rd right, with vertical lines in between
    \hline
     &\textbf{Euler}&\textbf{Second order Taylor}\\
    $t_n$&$y_n$&$y_n$\\
    \hline
    0&1&1\\
    0.25&1&0.96875\\
    0.5&1&0.947021484\\
    0.75&1&0.93162559\\
    1&1&0.920573828\\
    1.25&1&0.912567378\\
    1.5&1&0.906728956\\
    1.75&1&0.902451285\\
    2&1&0.899306321\\
    2.25&1&0.896988291\\
    2.5&1&0.895276597\\
    2.75&1&0.894010913\\
    3&1&0.893074081\\
    3.25&1&0.892380143\\
    3.5&1&0.89186584\\
    3.75&1&0.891484516\\
    4&1&0.891201702\\
    4.25&1&0.890991903\\
    4.5&1&0.890836242\\
    4.75&1&0.890720735\\
    5&1&0.890635016\\
    \hline
    \end{tabular}
  \end{center}
\end{table}

\newpage

\paragraph{b} $ty' = y\sin{t}$, $y(0)=2$
\subparagraph{} The equation can be rewritten as:
\[
    y' = \frac{y\sin{t}}{t}
\]

And to use second order Taylor, we need to find the
second derivative with respect of t:

\[
    y'' =  \frac{ y' \sin{(t)} -\cos{(t)}}{t} - \frac{y\sin{(t)}}{t^2}
\]

\begin{table}[ht]
  \begin{center}
    \label{tab:table 2.b}
    \begin{tabular}{|l|c|r|} % <-- Alignments: 1st column left, 2nd middle and 3rd right, with vertical lines in between
    \hline
     &\textbf{Euler}&\textbf{Second order Taylor}\\
    $t_n$&$y_n$&$y_n$\\
    \hline
    0&2&2\\
    0.25&2&2\\
    0.5&2.49480792&2.251295272\\
    0.75&2.97423346&2.601031357\\
    1&3.54108605&3.020058394\\
    1.25&4.16676884&3.489739458\\
    1.5&4.83885608&3.990376857\\
    1.75&5.53157792&4.498198696\\
    2&6.21177311&4.986261191\\
    2.25&6.84050431&5.426966153\\
    2.5&7.3775281&5.795464305\\
    2.75&7.78691323&6.073008809\\
    3&8.04288729&6.249396608\\
    3.25&8.1344614&6.323906695\\
    3.5&8.06752284&6.304603581\\
    3.75&7.86370622&6.206336666\\
    4&7.55630062&6.048056975\\
    4.25&7.18434609&5.85011344\\
    4.5&6.78653382&5.632024905\\
    4.75&6.39637189&5.410967685\\
    5&6.03943851&5.2009859\\
    \hline
    \end{tabular}
  \end{center}
\end{table}

\newpage

\paragraph{c} $y' = y(1+e^{2t})$, $y(0)=1$. To use second order Taylor, we need to find the
second derivative with respect of t:

\[
    y'' =  y' + y'e^{t} + ye^{t}
\]

\begin{table}[ht]
  \begin{center}
    \label{tab:table 2.c}
    \begin{tabular}{|l|c|r|} % <-- Alignments: 1st column left, 2nd middle and 3rd right, with vertical lines in between
    \hline
     &\textbf{Euler}&\textbf{Second order Taylor}\\
    $t_n$&$y_n$&$y_n$\\
    \hline
    0&1&1\\
    0.25&1.5&1.8125\\
    0.5&2.49327048&3.174121281\\
    0.75&4.81094105&5.505307351\\
    1&11.4039618&9.62422594\\
    1.25&35.3210806&17.21324543\\
    1.5&151.726064&31.95453893\\
    1.75&951.532443&62.49749471\\
    2&9067.02228&130.8520222\\
    2.25&135094.439&298.3238601\\
    2.5&3209071.5&753.9433018\\
    2.75&123078449&2150.747151\\
    3&7682923949&7048.455447\\
    3.25&7.8448E+11&26979.09032\\
    3.5&1.3143E+14&122434.6756\\
    3.75&3.6196E+16&667584.3523\\
    4&1.6406E+19&4424631.876\\
    4.25&1.2247E+22&36004622.09\\
    4.5&1.5063E+25&362771255.3\\
    4.75&3.0534E+28&4558091272\\
    5&1.0202E+32&71837960480\\
    \hline
    \end{tabular}
  \end{center}
\end{table}

\end{document}