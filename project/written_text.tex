\documentclass{article}

 \usepackage{amsthm}
 \usepackage{amsfonts}
 \usepackage{amssymb}
 \usepackage[margin=0.5in]{geometry}
 \usepackage[spanish, mexico]{babel}
 \usepackage[utf8]{inputenc}
 
 \title{Key aspects of the paper: Simulation of biochemical reactions with time-dependent rates by the rejection-based algorithm.}
 \author{Miguel A. Gomez B.}
 
 \begin{document}
 	\maketitle
 \paragraph{} Los autores Tahn y Priami en estudios previos que llevaron a cabo, construyeron algoritmos para la simulación de reacciones bioquímicas (como la replicación del ADN). Notan que aún se pueden realizar mejoras al algoritmo inicial, debido a que los procesos bioquímicos y físicos que subyacentes de la reacción se afectan moléculas presentes, que a su vez desencadenan otras reacciones químicas que dificultan la construcción de un modelo que sea computacionalmente eficiente. Los autores utilizan investigaciones realizadas en el pasado y agregan tres mejoras: rescribir las ecuaciones utilizadas en función del tiempo; incluir un algoritmo que rechaza los cambios de estado que no son posbiles dada una reacción; no utilizando el calcúlo de una integral en el cómputo. Los autores demuestran que su algoritmo es preciso y permite ser utilizado mas allá de la simulación de procesos bioquímicos y las mejoras realizadas al algoritmo le permiten obtener una mayor eficiencia.
 \end{document}