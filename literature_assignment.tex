\documentclass{article}

\usepackage{amsmath}
\usepackage{amsfonts}
\usepackage{amssymb}
\usepackage[utf8]{inputenc}

\title{Literature review assignment}
\author{Miguel A. Gomez B.}

\begin{document}
	
\maketitle
\paragraph{1} write a paragraph describing the key aspects of the papers listed below, the
paragraph should include:
\begin{itemize}
	\item 2 lines of motivation
	\item 2 lines for problem statement
	\item 2 lines for solution approach
	\item 2 lines for conclusions
\end{itemize}
\paragraph{2} List 10 papers and their abstracts related to the paper. This papers should be cited by the
paper selected above. Papers that cite the paper selected in the previous literal is also
recommended.

\section{Review}

New tools have been discovered to analize biological systems that give better and faster results. But many of these systems are complex, because of the multiple interactions at a macro-scale and micro-scale level. The approach consists on a deep analysis of the existing model, after that, they evaluate and propose the introduction of a stochastic process on the model to benefit of SHS theory. The authors expect run these models on computers: "The hope is to substitute costly and time-consuming in vitro or in vivo experiments or provide guidance for those experiments and generate better hypotheses."\cite{xiangfang_li_review_2017:1}

\section{Papers abstract related to the paper}

\paragraph{Transcriptional stochasticity in gene expression\cite{tomasz_lipniacki_transcriptional_2006}}

The article is focused on highlight and determine that there is an inherent stochasticity in the process of DNA transcription. Some interactions in the model between DNA, mRNA, regulatory proteins and gene expressions must admit stochastic effects. The authors construct a mathematical model to observe and verify these interactions by adding a stochastic component on current partial differential equations. The experimentation and results of the model conclude that there is a close relationship between a small set of mRNA and protein molecules that can be measured for a single cell. It is used on the main article as an evidence of the stochasticty of some biological processes such as DNA transcription.

\paragraph{Diversity in times of adversity: probabilistic strategies in microbial survival games\cite{denise_m._wolf_diversity_2005}}

the article explains that in some cases a probabilistic strategy is adopted by some organism such as \textit{Escherichia Coli} make to survive changes on their environment. The process that triggers this strategy it is too complex to model due to the broad range of changes at a macroscopic and microscopic level. The authors determine that it is better to use the approach from game theory and build a model that uses probabilities to determine wich strategy is used at a given environment. the authors conclude that it is a model that some proof that the alternation between losing strategies produce a wining strategy, but there is also a certain independence between the strategy used, which they assume it is to minimize growth rate at the expense of unfavorable conditions in the environment (the unlucky scenario). It is used on the main article to demonstrate that stochastisity plays an important role on biochemical processes.

\paragraph{In Silico, In Vitro, and In Vivo Studies Indicate the Potential Use of Bolaamphiphiles for Therapeutic siRNAs Delivery\cite{taejin_kim_silico_2013}}

RNA-based therapy technology to treat diseases such as cancer are making progress. Currently some sRNAs are being design to silence the expression of oncogenic paths, but this molecule has a short life on the blood serum, which makes it inefective for cancer treatment. The authors use properties of the Bolaamphiphiles to extend the live of the sRNAs in the blood making it effective for cancer treatment, to verify it, the authors use an arrangement of In silico (computer based), In vitro and In vivo studies that confirm the effectiveness of their method. It is used in the main article to demostrate that computer based model have been effective to analyze and predict results on biochemical reactions.

\paragraph{Intrinsic and extrinsic contributions to stochasticity in gene expression\cite{peter_s._swain_intrinsic_2002}}

\paragraph{motivation}
\paragraph{problem}
\paragraph{solution approach}
\paragraph{conclusions}
\paragraph{relationship with the article}

\paragraph{Diversity in times of adversity: probabilistic strategies in microbial survival games\cite{denise_m._wolf_diversity_2005}}

\paragraph{motivation}
\paragraph{problem}
\paragraph{solution approach}
\paragraph{conclusions}
\paragraph{relationship with the article}

\paragraph{Diversity in times of adversity: probabilistic strategies in microbial survival games\cite{denise_m._wolf_diversity_2005}}

\paragraph{motivation}
\paragraph{problem}
\paragraph{solution approach}
\paragraph{conclusions}
\paragraph{relationship with the article}

\paragraph{Diversity in times of adversity: probabilistic strategies in microbial survival games\cite{denise_m._wolf_diversity_2005}}

\paragraph{motivation}
\paragraph{problem}
\paragraph{solution approach}
\paragraph{conclusions}
\paragraph{relationship with the article}

\paragraph{Diversity in times of adversity: probabilistic strategies in microbial survival games\cite{denise_m._wolf_diversity_2005}}

\paragraph{motivation}
\paragraph{problem}
\paragraph{solution approach}
\paragraph{conclusions}
\paragraph{relationship with the article}

\paragraph{Diversity in times of adversity: probabilistic strategies in microbial survival games\cite{denise_m._wolf_diversity_2005}}

\paragraph{motivation}
\paragraph{problem}
\paragraph{solution approach}
\paragraph{conclusions}
\paragraph{relationship with the article}

\paragraph{Diversity in times of adversity: probabilistic strategies in microbial survival games\cite{denise_m._wolf_diversity_2005}}

\paragraph{motivation}
\paragraph{problem}
\paragraph{solution approach}
\paragraph{conclusions}
\paragraph{relationship with the article}

\bibliography{bibliography}
\bibliographystyle{ieeetr}

\end{document}